\documentclass[aspectratio=169]{beamer}

\definecolor{confirm-gray}{RGB}{126, 126, 126}

\mode<presentation>
{
    \usetheme{Madrid}
    \usecolortheme{default}
    \setbeamercovered{transparent}
}

\setbeamertemplate{navigation symbols}{}
\setbeamersize{text margin left=25pt, text margin right=25pt}

\usepackage[czech]{babel}
\usepackage[export]{adjustbox}

\title{Systém pro sportovní kluby}
\author{Lukáš Paukert}

\institute[]{
    Vedoucí práce: Ing. Filip Glazar\\
    \vspace{20pt}
    Fakulta informačních technologií\\
    České vysoké učení technické v Praze
}

% If you have a file called "university-logo-filename.xxx", where xxx
% is a graphic format that can be processed by latex or pdflatex,
% resp., then you can add a logo as follows:

% \pgfdeclareimage[height=0.5cm]{university-logo}{university-logo-filename}
% \logo{\pgfuseimage{university-logo}}


% If you wish to uncover everything in a step-wise fashion, uncomment
% the following command: 

%\beamerdefaultoverlayspecification{<+->}

\begin{document}

\begin{frame}
    \titlepage
\end{frame}

\begin{frame}{Motivace}
    \begin{figure}
        \includegraphics[width=0.9\linewidth, cfbox=lightgray 0.5pt 5pt]{images/kuoris.png}
    \end{figure}
\end{frame}

\begin{frame}{Zadání}
    \begin{itemize}
        \item navrhnout a implementovat webovou aplikaci pro sportovní kluby
        \item evidence a správa uživatelů a událostí
        \item propojení s informačním systémem ORIS
    \end{itemize}
\end{frame}

\begin{frame}{Technologie}
    \begin{itemize}
        \item výběr podřízen aktuálně využívaným hostingovým službám
        \item vybrané technologie
        \begin{itemize}
            \item PHP
            \item Symfony
            \item MariaDB
            \item Bootstrap
        \end{itemize}
    \end{itemize}
\end{frame}

\begin{frame}{Implementace}{ORIS API}
    \begin{itemize}
        \item komunikace pomocí \texttt{GET} požadavků s minimálně 2 query parametry
        \begin{itemize}
            \item \texttt{format}
            \item \texttt{method}
        \end{itemize}
        \item v současné době je podporováno 28 různých metod
        \item ve vytvořené aplikaci se využívají 3 metody
        \begin{itemize}
            \item \texttt{getEvent}
            \item \texttt{getEventList}
            \item \texttt{createEntry}
        \end{itemize}
    \end{itemize}
\end{frame}

\begin{frame}{Implementace}{Formuláře}
    \begin{figure}
        \includegraphics[width=0.7\linewidth, cfbox=lightgray 0.5pt 5pt]{images/categories.pdf}
    \end{figure}
\end{frame}

\begin{frame}{Implementace}{Responzivní design}
    \begin{figure}[h]
        \hfill
        \begin{minipage}[b]{0.3\linewidth}
            \includegraphics[width=0.9\linewidth, cfbox=lightgray 0.5pt 0pt]{images/homepage.pdf}
        \end{minipage}
        \hfill
        \begin{minipage}[b]{0.3\linewidth}
            \includegraphics[width=0.9\linewidth, cfbox=lightgray 0.5pt 0pt]{images/add-race.pdf}
        \end{minipage}
        \hfill
        \begin{minipage}[b]{0.3\linewidth}
            \includegraphics[width=0.9\linewidth, cfbox=confirm-gray 0.5pt 0pt]{images/confirm-dialog.pdf}
        \end{minipage}
        \hfill
    \end{figure}
\end{frame}

\begin{frame}{Závěr}
    \begin{itemize}
        \item všechny cíle byly splněny
        \item aplikace je připravena pro nasazení
        \item možná vylepšení
        \begin{itemize}
            \item e-mailové notifikace při přidání nové události
            \item pokročilejší propojení s IS ORIS
            \item zobrazování statistik pro trenéry a administrátory
        \end{itemize}
    \end{itemize}
\end{frame}

\begin{frame}
    Děkuji za pozornost.
\end{frame}

\appendix

% TODO
% \begin{frame}{Otázky oponenta}
%     Otázka
%     Odpověď
% \end{frame}

\end{document}
