\chapter{Nasazení}
Proces zprovoznění aplikace může být mnohdy poměrně náročný a problémy způsobené vzájemnou nekompatibilitou jednotlivých verzí nástrojů rozhodně dokáží tento proces velice znepříjemnit. Podobným komplikacím můžeme v dnešní době naštěstí předcházet například díky balíčkovacím systémům a nástroji Docker, který bude představen v podkapitole \ref{docker}. Následně bude také popsáno, jak lze právě s jeho pomocí spustit vytvořenou aplikaci a taktéž bude přiblíženo, jakým způsobem je možné aplikaci zprovoznit na současně využívaných hostingových službách.

\section{Docker}\label{docker}
V dnešní době je Docker využíván jak při vývoji, tak při samotném nasazování aplikací do~produkčního prostředí. Jedná se o nástroj, který poskytuje odlehčenou formu virtualizace. K tomu využívá takzvané kontejnery, které zahrnují samotný kód aplikace, všechny její závislosti a i její nastavení. Na rozdíl od klasických virtualizací tyto kontejnery neobsahují operační systém a přímo pomocí Docker engine komunikují s hostitelským operačním systémem. \cite{docker}

Díky této vlastnosti jsou kontejnery nástroje Docker méně náročné na hardware než klasické virtualizace. Mezi další důležité vlastnosti patří větší bezpečnost, poněvadž aplikace v různých kontejnerech jsou od sebe izolovány, a přenositelnost, která zaručuje, že aplikace budou fungovat na různých strojích s nainstalovaných nástrojem Docker vždy stejně. \cite{docker}

Ve chvíli, kdy se v rámci jednoho projektu využívá více různých kontejnerů, může s jejich správou pomoci nástroj Docker Compose, který se stará o jejich orchestraci. Jako hlavní konfigurační soubor využívá \mintinline{text}|docker-compose.yml|, v němž se definují a nastavují jednotlivé služby. Ty je následně možné spouštět a vypínat pomocí jediného příkazu.

\section{Požadovaný software}
Pro zprovoznění aplikace dle podstupu uvedeného v podkapitole \ref{deploy-steps} je zapotřebí mít nainstalován následující software:
\begin{itemize}
	\item Docker (popsán v předchozí sekci \ref{docker})
	\item Docker Compose (nadstavba nad nástrojem Docker, taktéž popsána v sekci \ref{docker})
\end{itemize}

\section{Postup pro zprovoznění}\label{deploy-steps}
Díky nástroji Docker je zprovoznění aplikace velice jednoduché. Po~provedení níže uvedených kroků bude systém dostupný na adrese \mintinline{text}|https://localhost/|.

\begin{enumerate}
	\item sestavení kontejnerů příkazem \mintinline{text}|docker-compose build --pull --no-cache|
	\begin{itemize}
		\item instalace závislostí pomocí balíčkovacího systému Composer proběhne automaticky (v případě potřeby lze spustit i pomocí \mintinline{text}|docker-compose exec php php composer install|)
	\end{itemize}
	\item spuštění kontejnerů pomocí příkazu \mintinline{text}|docker-compose up -d|
	\begin{itemize}
		\item databázové migrace se spustí také automaticky (případně je lze spustit i ručně pomocí \mintinline{text}|docker-compose exec php php bin/console doctrine:migrations:migrate|)
		\item v rámci migrací bude do databáze vložen administrátor se jménem \mintinline{text}|KUO9801| a heslem \mintinline{text}|KUO9801| a i několik dat pro číselníky disciplín a úrovní
	\end{itemize}
	\item testovací data obsahující další členy a události včetně přihlášek lze vygenerovat pomocí příkazu \mintinline{text}|docker-compose exec php php bin/console doctrine:fixtures:load --append|
	\item všechny kontejnery lze naopak zastavit pomocí \mintinline{text}|docker-compose down --remove-orphans|
\end{enumerate}

\section{Webhosting}
TODO
