\chapter{Nasazení}
Proces zprovoznění aplikace může být mnohdy poměrně náročný a problémy způsobené vzájemnou nekompatibilitou jednotlivých verzí nástrojů rozhodně dokáží tento proces velice znepříjemnit. Podobným komplikacím můžeme v dnešní době naštěstí předcházet například díky balíčkovacím systémům a nástroji Docker, který bude představen v podkapitole \ref{docker}. Následně bude také popsáno, jak lze právě s jeho pomocí spustit vytvořenou aplikaci a taktéž bude přiblíženo, jakým způsobem je možné aplikaci zprovoznit na současně využívaných hostingových službách.

\section{Docker}\label{docker}
V dnešní době je Docker využíván jak při vývoji, tak při samotném nasazování aplikací do~produkčního prostředí. Jedná se o nástroj, který poskytuje odlehčenou formu virtualizace. K tomu využívá takzvané kontejnery, které zahrnují samotný kód aplikace, všechny její závislosti a i její nastavení. Na rozdíl od klasických virtualizací tyto kontejnery neobsahují operační systém a přímo pomocí Docker engine komunikují s hostitelským operačním systémem. \cite{docker}

Díky této vlastnosti jsou kontejnery nástroje Docker méně náročné na hardware než klasické virtualizace. Mezi další důležité vlastnosti patří větší bezpečnost, poněvadž aplikace v různých kontejnerech jsou od sebe izolovány, a přenositelnost, která zaručuje, že aplikace budou fungovat na různých strojích s nainstalovaných nástrojem Docker vždy stejně. \cite{docker}

Ve chvíli, kdy se v rámci jednoho projektu využívá více různých kontejnerů, může s jejich správou pomoci nástroj Docker Compose, který se stará o jejich orchestraci. Jako hlavní konfigurační soubor využívá \mintinline{text}|docker-compose.yml|, v němž se definují a nastavují jednotlivé služby. Ty je následně možné spouštět a vypínat pomocí jediného příkazu.

\section{Požadovaný software}
Pro zprovoznění aplikace dle podstupu uvedeného v podkapitole \ref{deploy-steps} je zapotřebí mít nainstalován následující software:
\begin{itemize}
	\item Docker (popsán v předchozí sekci \ref{docker})
	\item Docker Compose (nadstavba nad nástrojem Docker, taktéž popsána v sekci \ref{docker})
\end{itemize}

\section{Postup pro zprovoznění}\label{deploy-steps}
Díky nástroji Docker je zprovoznění aplikace velice jednoduché. Po~provedení níže uvedených kroků bude systém dostupný na adrese \mintinline{text}|https://localhost/|.

\begin{enumerate}
	\item sestavení kontejnerů příkazem \mintinline{text}|docker-compose build --pull --no-cache|
	\begin{itemize}[topsep=0pt]
		\item instalace závislostí pomocí balíčkovacího systému Composer proběhne automaticky
	\end{itemize}
	\item spuštění kontejnerů pomocí příkazu \mintinline{text}|docker-compose up -d|
	\begin{itemize}[topsep=0pt]
		\item databázové migrace se spustí také automaticky
		\item v rámci migrací bude do databáze vložen administrátor se jménem \mintinline{text}|KUO9801| a heslem \mintinline{text}|KUO9801| a i několik dat pro číselníky disciplín a úrovní
	\end{itemize}
	\item testovací data obsahující další členy a události včetně přihlášek lze vygenerovat pomocí příkazu \mintinline{text}|docker-compose exec php php bin/console doctrine:fixtures:load --append|
	\item všechny kontejnery lze naopak zastavit pomocí \mintinline{text}|docker-compose down --remove-orphans|
\end{enumerate}

\section{Webhosting}
Již od počátku bylo cílem vyvinout aplikaci, která by se mohla provozovat na aktuálně využívaných hostingových službách, jakými jsou sdílený webhosting a databázový server MariaDB od~společnosti WEDOS. Z tohoto důvodu byl i specifikován nefunkční požadavek \ref{n:technologies}, který tento cíl reflektoval. S provozem webové aplikace na těchto službách, však přichází řada úskalí, neboť k webhostingu například není možné přistupovat přes SSH protokol. Jenom toto omezení způsobuje řadu komplikací, poněvadž je tímto například znemožněno instalovat závislosti pomocí balíčkovacího systému Composer nebo spouštět standardní způsobem databázové migrace.

Pro zprovoznění vytvořené aplikace na produkčním serveru bylo tedy třeba provést několik dalších úprav a nastavení. Prvním krokem bylo přidání závislosti \mintinline{text}|symfony/apache-pack|, jež zajistila, že systém bude fungovat na webovém serveru Apache, který využívají sdílené webhostingy od společnosti WEDOS. S ohledem na omezení, která jsou kladena na tyto webhostingy, muselo být z vygenerovaného \mintinline{text}|.htaccess| souboru odebráno nepodporované nastavení, jež je na~výpisu kódu \ref{symfony-htaccess}. \cite{symfony_wedos}

\begin{listing}[h]
    \caption{Nepodporované nastavení ve vygenerovaném .htaccess souboru}\label{symfony-htaccess}
    \begin{minted}{apacheconf}
<IfModule mod_negotiation.c>
    Options -MultiViews
</IfModule>
    \end{minted}
\end{listing}

\vspace{-2mm}

Další úpravy kódu bylo třeba udělat ve spojení s voláním PHP funkce \mintinline{text}|putenv()|, která je na~webhostingu zakázána. Poslední nutné nastavení spočívalo ve vytvoření dalšího \mintinline{text}|.htaccess| souboru s obsahem uvedeném na výpisu kódu \ref{redirect-htaccess} v kořenové složce projektu. Pravidlo uvedené v tomto souboru zajišťuje přesměrování požadavků na přístupový bod naší aplikace, kterým je soubor \mintinline{text}|index.php| umístěný ve složce \mintinline{text}|public| \cite{symfony_wedos}.

\begin{listing}[h]
    \caption{Obsah .htaccess souboru umístěného v kořenové složce projektu}\label{redirect-htaccess}
    \begin{minted}{apacheconf}
RewriteEngine On
RewriteCond %{REQUEST_URI} !^public
RewriteRule ^(.*)$ public/$1 [L]
    \end{minted}
\end{listing}

Jak již bylo zmíněno, k webhostingu není možné přistupovat přes SSH protokol. Z tohoto důvodu je na něj nutné nahrát všechny soubory vytvořené aplikace včetně jejích závislostí ručně přes FTP protokol. Posledním nutným úkonem je vytvoření odpovídající struktury v databázi a přidání dat pro číselníky. Toho může být docíleno například spuštěním databázových migrací v lokálním prostředí, vytvořením zálohy této databáze a následným importováním této zálohy do produkčního prostředí.
