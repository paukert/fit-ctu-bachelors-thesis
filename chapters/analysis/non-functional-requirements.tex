\section{Nefunkční požadavky}
Nefunkční požadavky se na~rozdíl od~funkčních požadavků zabývají vlastnostmi a~omezeními daného systému. Mohou se mezi~nimi objevit požadavky, které budou podstatné jak~pro~samotné uživatele systému, jako například doba odezvy, zabezpečení systému, způsob zobrazování na~mobilních zařízeních, tak zde mohou být uvedeny i~požadavky techničtějšího charakteru, jako~například použití konkrétních technologií nebo~možnost snadné rozšiřitelnosti systému. \cite{requirements}

\begin{enumerate}[label=\textcolor{decoration}{\textbf{N\arabic*}}, leftmargin=7mm]
	\myItem{responzivita}
	Webová aplikace bude přizpůsobovat svůj vzhled na~základě rozlišení klientského zařízení.

	\myItem{lokalizace}
	Aplikace by měla být připravena na~přeložení do~dalších jazyků.

	\myItem{technologie\label{n:technologies}}
	S~ohledem na~aktuálně využívané hostingové služby musí být aplikace napsána v~programovacím jazyce PHP a~pro~ukládání dat musí být využita relační databáze MariaDB.

	\myItem{zabezpečení}
	Komunikace s~aplikací bude probíhat pouze přes~šifrovaný protokol HTTPS.

	\myItem{rozšiřitelnost a~modifikovatelnost}
	Aplikace by měla být snadno rozšiřitelná a~modifikovatelná. Její využití by tedy nemělo být limitováno pouze na~kluby orientačního běhu, ale své uplatnění by měla nalézt i~u~sportovních organizací zabývajících se jiným sportem.

	\myItem{výkon~a dostupnost}
	Aplikace bude schopná v~jednu chvíli obsluhovat jednotky až~nižší desítky uživatelů s~odezvou v~řádu sekund.
\end{enumerate}
