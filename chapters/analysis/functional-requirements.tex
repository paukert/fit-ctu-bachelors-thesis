\section{Funkční požadavky}
Funkční požadavky popisují požadované funkcionality systému. Mohou například specifikovat, jakým způsobem bude uživatel moci pracovat se systémem, jaké procesy by měl systém podporovat a~jaké budou vstupy a~výstupy těchto procesů. \cite{requirements}

\begin{enumerate}[label=\textcolor{decoration}{\textbf{F\arabic*}}, leftmargin=6mm]
	\myItem{evidence a~správa uživatelů}
	Systém bude umožňovat registraci nových uživatelů, neregistrovaný uživatel nemá přístup do~systému. Při~registraci bude nutno zadat registrační číslo, e-mailovou adresu, heslo a~jméno a~příjmení člena klubu. Právě pomocí registračního čísla a~zadaného hesla bude následně uživatel přistupovat do~aplikace. Kromě těchto základních údajů bude mít administrátor možnost uživateli nastavovat a~upravovat stav osobního konta, typ členství (aktivní nebo~pasivní) a~roli (viz~sekce \ref{section:role}).

	\myItem{evidence a~správa závodů a~tréninků\label{f:events}}
	Dále systém bude umožňovat evidenci a~správu tréninků a~závodů (událostí). Konkrétně je potřeba mít možnost u~každé události evidovat:
	\begin{itemize}
		\item název
		\item typ události (trénink nebo~závod)
		\item datum a~čas události (kdy událost začíná)
		\item místo konání
		\item pořadatel
		\item datum a~čas uzávěrky přihlášek
		\item disciplínu
		\item vypsané kategorie
		\item další informace
		\item v~případě závodu bude možné dále evidovat:
		\begin{itemize}
			\item typ závodu
			\item webovou stránku
		\end{itemize}
		\item u~tréninku bude možné oproti závodu naopak evidovat maximální kapacitu
	\end{itemize}

	Samotné zadání události do~systému bude umožněno ručním vyplněním všech požadovaných údajů nebo~automatickým importem z~informačního systému Českého svazu orientačních sportů pomocí unikátního identifikátoru závodu.

	Na~hlavní stránce systému by se měly zobrazovat závody a tréninky s~nejbližší uzávěrkou přihlášek. V~zadaných trénincích a~závodech bude možné vyhledávat a~filtrovat a~zadanou událost musí být možno zrušit bez~odstraňování ze~systému. Taková událost bude následně od~nezrušených událostí vizuálně odlišena a~nebude možné se na~ni dále přihlašovat.

	\myItem{přihlašování na~závody a~tréninky}
	Na~událost evidovanou v~systému se mohou registrovaní uživatelé přihlásit, pokud dosud neproběhla uzávěrka přihlášek. Při~přihlášení si uživatelé musí navíc vybrat jednu z~nabízených kategorií, do~které se chtějí přihlásit, a~mohou volitelně sdělit, zdali mají možnost jet vlastním autem a~svést s~sebou i~další členy klubu. Svou účast mohou uživatelé do~konce uzávěrky přihlášek také odřeknout.

	\myItem{synchronizace přihlášek}
	Trenéři klubu budou mít možnost odeslat všechny přihlášky evidované u konkrétního závodu do~IS~ORIS přes~jeho API. Tato možnost bude ze~zřejmých důvodů dostupná pouze pro~závody, které jsou v~IS~ORIS evidované.

	\myItem{oznámení}
	Trenéři a~administrátoři systému budou mít možnost vytvořit oznámení, která se budou zobrazovat na~hlavní stránce systému.

	\myItem{komentáře}
	Komentáře mohou psát registrovaní uživatelé ke~všem tréninkům a~závodům. Nejnovější komentáře by se měly zobrazovat (včetně informace kam byly napsány) na~hlavní stránce systému.
\end{enumerate}

\subsection{Role}
\label{section:role}
Každému uživateli bude náležet právě jedna z~následujících uživatelských rolí:
\begin{itemize}
	\item registrovaný uživatel
	\begin{itemize}[topsep=0pt]
		\item základní uživatelská role, která nemá žádná speciální oprávnění
		\item výčet akcí, které může registrovaný uživatel vykonávat:
		\begin{itemize}
			\item přihlašovat a~odhlašovat se z~tréninků a~závodů
			\item psát komentáře k~jednotlivým tréninkům a~závodům a~následně je i~upravovat
			\item měnit svou e-mailovou adresu a~heslo pro~přihlášení do~systému
		\end{itemize}
	\end{itemize}
	\item trenér
	\begin{itemize}[topsep=0pt]
		\item role, která náleží trenérům působícím v~klubu
		\item oproti roli registrovaný uživatel má právo:
		\begin{itemize}
			\item přidávat a~následně spravovat tréninky a~závody
			\item odesílat evidované přihlášky do~IS~ORIS
			\item psát a~upravovat svá oznámení
		\end{itemize}
	\end{itemize}
	\item administrátor
	\begin{itemize}[topsep=0pt]
		\item má kompletní přístup ke~všem funkcionalitám systému
		\item oproti uživatelům s~rolí trenér může navíc:
		\begin{itemize}
			\item upravovat a~mazat veškeré komentáře a oznámení
			\item spravovat uživatele
			\item odhlašovat a~přihlašovat uživatele na~tréninky a~závody (i~po~termínu přihlášek)
			\item odebírat a~přidávat informaci o~možnosti vzít auto (i~po~termínu přihlášek)
		\end{itemize}
	\end{itemize}
\end{itemize}
