\section{Funkční požadavky}

\begin{enumerate}[label=\textcolor{decoration}{\textbf{F\arabic*}}]
	\myItem{evidence a~správa uživatelů}
	Systém bude umožňovat registraci nových uživatelů, neregistrovaný uživatel nemá přístup do~systému. Při~registraci bude nutno zadat minimálně e-mailovou adresu, heslo a~jméno a~příjmení člena klubu. Právě pomocí e-mailové adresy a~zadaného hesla bude následně uživatel přistupovat do~systému.

	Kromě těchto základních údajů bude mít administrátor možnost nastavovat a~upravovat uživatelovo registrační číslo, stav osobního konta, typ členství (aktivní nebo~pasivní) a~roli (viz~sekce \ref{section:role}).

	\myItem{evidence a~správa závodů a~tréninků}
	Dále systém bude umožňovat evidenci a~správu tréninků a~závodů (událostí). Konkrétně je potřeba mít možnost u~každé události evidovat:
	\begin{itemize}
		\item název
		\item typ události (trénink nebo~závod)
		\item datum a~čas události (od~kdy se začíná startovat~/~kdy událost začíná)
		\item místo konání
		\item pořadatel
		\item uzávěrka přihlášek
		\item nepovinně dále:
		\begin{itemize}
			\item typ závodu (závod může být zařazen do~více soutěží; bude možné nastavit, pouze pokud se jedná o~závod)
			\item disciplína
			\item vypsané kategorie
			\item webová stránka
			\item další informace
		\end{itemize}
	\end{itemize}

	Samotné zadání události do~systému bude umožněno ručním vyplněním všech požadovaných údajů nebo~automatickým importem z~informačního systému Českého svazu orientačních sportů ORIS (dále jen „IS~ORIS“) pomocí unikátního identifikátoru závodu. V~případě importu z~IS~ORIS bude možné též provést automatickou aktualizaci údajů, které se od~doby importu v~IS~ORIS změnily.

	Na~hlavní stránce systému by se měly zobrazovat závody s~nejbližší uzávěrkou přihlášek a~nejbližší závody, na~které je daný uživatel přihlášený. V~zadaných trénincích a~závodech bude možné vyhledávat a~filtrovat a~zadanou událost musí být možno zrušit bez~odstraňování ze~systému (taková událost bude následně vizuálně od~nezrušených událostí odlišena a~nebude možné se na~ni dále přihlašovat).

	\myItem{přihlašování na~závody a~tréninky}
	Na~událost evidovanou v~systému se mohou registrovaní uživatelé přihlásit, pokud dosud neproběhla uzávěrka přihlášek. Při~přihlášení si uživatelé musí navíc vybrat jednu z~nabízených kategorií, do~které se chtějí přihlásit, a~mohou volitelně sdělit, zdali mají možnost jet vlastním autem a~svést s~sebou i~další členy klubu.

	\myItem{synchronizace přihlášek}
	Po~uzávěrce přihlášek bude mít trenér možnost odeslat všechny evidované přihlášky do~IS ORIS přes~jeho API. Tato možnost bude samozřejmě dostupná pouze pro~závody, které jsou v~IS~ORIS evidované.

	\myItem{e-mailová upozornění}
	Systém bude umožňovat při~přidávání události odeslat informační e-mail se základními informacemi o~dané události. E-mail bude odeslán pouze těm registrovaným uživatelům, kteří si ve~svém nastavení povolili zasílání e-mailových upozornění.

	\myItem{oznámení}
	Trenéři a~administrátoři systému budou mít možnost vytvořit oznámení, která se budou zobrazovat na~hlavní stránce systému.

	\myItem{komentáře}
	Komentáře mohou psát registrovaní uživatelé ke~všem tréninkům, závodům, oznámením a~také do~globálního „chatu“. Nejnovější komentáře by se měly zobrazovat (včetně informace kam byly napsány) na~hlavní stránce systému.
\end{enumerate}

\subsection{Role}
\label{section:role}
Každému uživateli bude náležet právě jedna z~následujících uživatelských rolí:
\begin{itemize}
	\item registrovaný uživatel
	\begin{itemize}
		\item základní uživatelská role, která nemá žádná speciální oprávnění
		\item výčet akcí, které může registrovaný uživatel vykonávat:
		\begin{itemize}
			\item přihlašovat a~odhlašovat se z~tréninků a~závodů
			\item psát komentáře k~jednotlivým tréninkům, závodům, oznámením a~do~globálního „chatu“ a~následně je i~upravovat
			\item měnit svou e-mailovou adresu a~heslo pro~přihlášení do~systému
			\item možnost nastavit si e-mailová upozornění na~nově přidané události do~systému
		\end{itemize}
	\end{itemize}
	\item trenér
	\begin{itemize}
		\item role, která náleží trenérům působícím v~klubu
		\item oproti roli registrovaný uživatel má právo:
		\begin{itemize}
			\item přidávat a~následně spravovat tréninky a~závody
			\item psát a~upravovat svá oznámení
		\end{itemize}
	\end{itemize}
	\item administrátor
	\begin{itemize}
		\item má kompletní přístup ke~všem funkcionalitám systému
		\item oproti uživatelům s~rolí trenér může navíc:
		\begin{itemize}
			\item upravovat a~mazat veškeré komentáře, oznámení, tréninky a~závody
			\item registrovat a~spravovat uživatele
			\item odhlašovat a~přihlašovat uživatele na~tréninky a~závody (i~po~termínu přihlášek)
			\item odebírat a~přidávat informaci o~možnosti vzít auto (i~po~termínu přihlášek)
		\end{itemize}
	\end{itemize}
\end{itemize}
