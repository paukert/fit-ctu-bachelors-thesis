\section{Architektura}
Jedno z~prvních rozhodnutí, které bylo v~souvislosti s~návrhem aplikace potřeba udělat, se týkalo architektury aplikace. Dle \cite{twa_architecture} se v~případě webových aplikací můžeme setkat s~rozdělením na~thin-client a~thick-client. První zmíněná možnost znamená, že veškerá logika se nachází na~straně serveru. Pokud se aplikace na~straně serveru neskládá z~více samostatně fungujících částí, můžeme se také setkat s~označením monolitická aplikace. U~možnosti thick-client je naopak aplikace na~straně klienta a~server poskytuje pouze API (viz sekce \ref{implementation:api}) pro~účely komunikace.

Monolitické aplikace nejsou v~dnešní době již tak populární, jako tomu bylo v minulosti, nicméně stále mají své uplatnění. Typicky se monolitická architektura využívá u~malých aplikací, u~kterých se v~budoucnu nepředpokládá větší rozšiřování. Jednou z~výhod je jistě fakt, že monolitické aplikace se rychleji vyvíjejí a~také se jednodušeji testují. Jejich nevýhodou může být naopak složitější údržba a~problémové škálování. \cite{twa_architecture, monolithic_architecture} Jelikož je vyvíjená aplikace zamýšlená pouze pro~jednotlivé sportovní kluby a~nepředpokládá se potřeba přistupovat a~upravovat data jiným způsobem než z~nové webové aplikace, tak jsem se i~s~ohledem na~rychlejší vývoj rozhodl využít monolitickou architekturu.

Výběr této architektury však neznamená, že by nemohla být aplikace vnitřně žádným způsobem členěna. Nějaká forma rozdělení je naopak i žádoucí, aby se oddělila odpovědnost a vznikla určitá forma abstrakce nad jednotlivými částmi systému. Díky tomu je například možné vyměnit datové úložiště bez dalších (nebo s minimálním počtem) změn v částech, které se zabývají uživatelským rozhraním nebo obsahují samotnou logiku aplikace. Jednou z možností takového strukturování je využití architektury MVC, která je podrobněji představena v sekci \ref{section:mvc}.

\subsection{MVC architektura}\label{section:mvc}
Následující část textu vychází ze zdrojů \cite{it_network_mvc, mdn_mvc}. Architektura model–view–controller odděluje datový model aplikace od řídící logiky a prezentační vrstvy s uživatelským rozhraním. Nejvíce se její použití rozšířilo u webových aplikací, poněvadž je součástí i mnoha webových frameworků. Na MVC architektuře jsou například založeny frameworky Laravel a Symfony, ASP.NET MVC framework a mnoho dalších.

MVC architektura se tedy skládá ze tří částí. Vrstva modelů definuje s jakými daty bude aplikace pracovat. Může také obsahovat validační pravidla pro tato data, případně i další funkce, které se váží k danému modelu. Takovým příkladem by mohla být funkce na výpočet věku z informace o datu narození. Při využití objektově relačního mapování (ORM) kopírují modely z velké části strukturu tabulek v databázi. Jeden model typicky představuje konkrétní tabulku a jeho atributy představují sloupce dané tabulky.

Pohledy se starají o zobrazení dat uživateli. V případě webových aplikací bývá součástí pohledu šablona obsahující značkovací jazyk HTML, do které jsou následně vkládána data z modelů.

Poslední částí jsou controllery, jež se starají o načítání dat z modelů a jejich aktualizaci v závislosti na vstupu od uživatele. Propojují všechny ostatní vrstvy do funkčního celku.

\begin{figure}[h]
	\caption{Životní cyklus požadavku}
	\label{figure:mvc}
	\centering
	\includegraphics[width=0.9\textwidth]{images/mvc.pdf}
\end{figure}

Obrázek \ref{figure:mvc} zobrazuje typický životní cyklus požadavku. Můžeme si například představit situaci, kdy si chce uživatel zobrazit nadcházející události evidované v aplikaci. Požadavek od~uživatele nejdříve zachytí část systému nazývaná jako router, která na základě URL adresy rozpozná, jakému controlleru požadavek předat. Daný controller si následně od konkrétního modelu vyžádá uložená data o nadcházejících událostech. Poté, co je od něj získá, předá tato data pohledu, který je již pouze vloží do připravené šablony. Sestavená stránka je nakonec zaslána zpět uživateli.
