\section{Technologie}
Další důležité rozhodnutí se při návrhu webové aplikace týká zvoleného programovacího jazyka, neboť je možné vybírat z mnoha možností a každý jazyk má své specifické vlastnosti. Dle W3Techs převládá mezi stránkami, u nichž je znám programovací jazyk na~straně serveru, již minimálně od roku 2011 programovací jazyk PHP. Z dostupných dat k 1. dubnu 2022 používá PHP více než 77~\% webových stránek. Mezi další využívané jazyky se následně řadí ASP.NET (7,8 \%), Ruby (6 \%) a Java (3,9 \%). \cite{php_w3techs} S ohledem na nefunkční požadavek \ref{n:technologies}, který vyžaduje vytvoření aplikace v programovacím jazyku PHP, bude pro~vývoj využit právě tento jazyk.

\subsection{PHP}
 PHP je skriptovací programátorský jazyk, který je určen především pro~vývoj webových aplikací. Nejčastěji je PHP kód spouštěn na~webovém serveru na~základě požadavku od~klienta, kterým může být například webový prohlížeč. Po~vykonání kódu je výsledek zaslán zpět klientovi. \cite{php_intro_1, php_intro_2} Nejnovější stabilní verze PHP (ke~dni 5.~2.~2022) je verze 8.1 \cite{php_version}, avšak pro~vývoj této aplikace bude z~důvodů kompatibility využita verze 8.0.
 
 Podobně jako v jiných programovacích jazycích, tak i pro jazyk PHP byla vytvořena celá řada frameworků a knihoven, které by měly usnadnit vytváření nových aplikací. Dle průzkumu za rok 2021 se mezi nejpoužívanější PHP frameworky řadí Laravel a Symfony \cite{php_jetbrains}.
 
\subsection{Symfony}
Symfony se řadí mezi nejstarší PHP frameworky, neboť se prvního vydání dočkalo již v roce 2005 \cite{symfony_legacy}. Od té doby se však neustále vyvíjí a v roce 2021 se dle průzkumu od JetBrains jednalo po Laravelu o druhý nejpoužívanější PHP framework \cite{php_jetbrains}. Symfony však není pouze samotný framework, ale je to sada znovupoužitelných komponent. To dokazuje i skutečnost, že Laravel a mnoho dalších projektů některé Symfony komponenty využívá. Mezi takové projekty se například řadí známé systémy pro správu obsahu (CMS) Drupal a Joomla! nebo řešení pro~internetové obchody PrestaShop. \cite{symfony_projects}

Pokud bychom porovnávali Laravel a Symfony, tak zjistíme, že to jsou v základu podobné frameworky a liší se spíše v drobnostech. Oba jsou založeny na MVC architektuře, v obou jsou k dispozici nástroje pro objektově relační mapování, šablonovací procesory a podobně. Zásadnější rozdíly bychom například mohli nalézt v možnostech konfigurace. Symfony v porovnání s Laravelem umožňuje pokročilejší nastavení, díky čemuž může být však obtížnější na naučení a vývoj aplikace většinou zabere i více času. U větších projektů však tyto důsledky ustupují do pozadí a vyvíjená aplikace může být přizpůsobena přesně dne potřeb cílových skupin. \cite{symfony_laravel_comparison}

Jedním z důvodů, proč jsem si pro implementaci této aplikace vybral Symfony, je skutečnost, že Laravel využívá „magii“ (například magickou metodu \mintinline{php}|__get()| pro přístup k atributům), která stěžuje statickou analýzu kódu \cite{larastan}. Dalším faktorem byla již zmíněná větší volnost při využívání Symfony frameworku a také fakt, že s používáním Symfony jsem měl již nějaké zkušenosti.

\subsection{Databáze}
Pro ukládání dat lze využít několik typů databází. V současnosti jsou stále nejpopulárnější relační databáze, které jsou občas z důvodu využívaného dotazovacího jazyka také označovány jako SQL databáze. Jejich základem jsou tabulky, které mají pevně danou strukturu. Jednotlivé řádky tabulky  představují uložené záznamy a sloupce obsahují vlastnosti daných záznamů. \cite{databases} Dnes se však kromě relačních databází můžeme setkat i s NoSQL databázemi, které například mají své uplatnění při ukládání nestrukturovaných nebo velkých dat \cite{nosql}.

Vyvíjená aplikace bude pracovat s pevně struktorovanými daty, a proto bude využita relační databáze. Vzhledem k nefunkčnímu požadavku \ref{n:technologies} se bude jednat o databázi MariaDB, která vznikla v roce 2009 jako kopie MySQL kvůli obavám jejího dalšího směřování po odkoupení společností Oracle \cite{mariadb}.

\subsection{Doctrine}
Jednou z dalších zvolených technologií pro vývoj nové aplikace je Doctrine. Výběr Doctrine je úzce spojen s výběrem frameworku, neboť Symfony přímo poskytuje možnost integrace s touto komponentou. Jedná se o sadu knihoven, které umožňují objektově relační mapování (ORM) a které nabízí objektově orientované API (a řadu dalšího) pro manipulaci s daty a se strukturou databáze \cite{doctrine_orm, doctrine_dbal}.

Při vývoji aplikací v objektově orientovaných jazycích se často využívají objekty, které nám sdružují vlastnosti předmětů podobně jako tomu je v reálném světě. Tento přístup k datům však neodpovídá tomu, jakým způsobem se data ukládají v relačních databázích. V nich je jeden konkrétní objekt reprezentován jako jeden řádek s odpovídajícím počtem sloupců. Rozdíly mezi těmito přístupy vyrovnává právě ORM, které zajišťuje automatickou konverzi dat mezi objektově orientovaným jazykem a relační databází. \cite{doctrine_orm}

Další částí Doctrine je již zmíněná abstraktní vrstva nad databází (DBAL). Díky ní nemusíme k databázi přistupovat napřímo, ale můžeme využít připravené metody, které poskytují jednotný přístup k datům a struktuře databáze nezávisle na využívaném řešení. V současnosti jsou podporovány všechny nejpoužívanější typy relačních databází zahrnující MySQL, Oracle, PostgreSQL, SQLite a Microsoft SQL Server. \cite{doctrine_dbal}

\subsection{Bootstrap}
Poslední technologie, která bude představena v této kapitole, je Bootstrap. Jedná se svobodný a otevřený software, který usnadňuje tvorbu uživatelského rozhraní ve~webových aplikacích. Zahrnuje šablony napsané v HTML, CSS a další rozšíření implementované v jazyku JavaScript. Šablony jsou k dispozici pro všechny základní HTML elementy a umožňují jednoduše vytvářet responzivní uživatelské rozhraní. \cite{bootstrap}
