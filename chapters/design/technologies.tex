\section{Technologie}
Další důležité rozhodnutí se při návrhu webové aplikace týká zvoleného programovacího jazyka, neboť je možné vybírat z mnoha možností a každý jazyk má své specifické vlastnosti. Dle W3Techs převládá mezi stránkami, u nichž je znám programovací jazyk na~straně serveru, již minimálně od roku 2011 programovací jazyk PHP. Z dostupných dat k 1. dubnu 2022 používá PHP více než 77~\% webových stránek. Mezi další využívané jazyky se následně řadí ASP.NET (7,8 \%), Ruby (6 \%) a Java (3,9 \%). \cite{php_w3techs} S ohledem na nefunkční požadavek \ref{n:technologies}, který vyžaduje vytvoření aplikace v programovacím jazyku PHP, bude pro~vývoj využit právě tento jazyk.

\subsection{PHP}
 PHP je skriptovací programátorský jazyk, který je určen především pro~vývoj webových aplikací. Nejčastěji je PHP kód spouštěn na~webovém serveru na~základě požadavku od~klienta, kterým je velice často internetový prohlížeč. Po~vykonání kódu je výsledek zaslán zpět klientovi. \cite{php_intro_1, php_intro_2} Nejnovější stabilní verze PHP (ke~dni 5.~2.~2022) je verze 8.1\cite{php_version}, avšak pro~vývoj této aplikace bude z~důvodů kompatibility využita verze 8.0.
 
 Podobně jako v jiných programovacích jazycích, tak i pro jazyk PHP byla vytvořena celá řada frameworků a knihoven, které by měly usnadnit vytváření nových aplikací. Dle průzkumu za rok 2021 se mezi nejpoužívanější PHP frameworky řadí Laravel a Symfony \cite{php_jetbrains}.
 
\subsection{Symfony}
Symfony se řadí mezi nejstarší PHP frameworky, neboť se prvního vydání dočkalo již v roce 2005 \cite{symfony_legacy}. Od té doby se však neustále vyvíjí a na podzim roku 2021 byla vydána již šestá major verze \cite{symfony_6}. V roce 2021 se podle průzkumu od JetBrains jednalo po Laravelu o druhý nejpoužívanější PHP framework\cite{php_jetbrains}. Symfony však není pouze samotný framework, ale je to sada znovupoužitelných komponent. To dokazuje i fakt, že framework Laravel využívá i několik Symfony komponent a není rozhodně sám. Mezi další projekty, které jsou založené na Symfony nebo využívají Symfony komponenty se například řadí Drupal, Joomla!, PrestaShop a mnoho dalších.\cite{symfony_projects}

Pokud bychom porovnávali Laravel a Symfony, tak zjistíme, že to jsou poměrně podobné frameworky a liší se spíše v drobnostech. Oba jsou založeny na MVC architektuře, v obou jsou k dispozici nástroje pro objektově relační mapování, šablonovací procesory atd. Rozdíly bychom například mohli nalézt v možnostech konfigurace. Symfony v porovnání s Laravelem umožňuje detailnější nastavení, díky čemuž je těžší na naučení a vývoj aplikace většinou zabere i více času.\cite{symfony_laravel_comparison}


... Jedním z dalších faktorů, který mě vedl pro výběr Symfony frameworku, je fakt, že jsem s ním již v minulosti pracoval.

\subsection{Doctrine}
TODO

\subsection{MariaDB}
TODO

\subsection{Boostrap}
TODO
