\chapter{Implementace}
Tato kapitola se zabývá procesem implementace nové webové aplikace. Nejdříve budou představeny nástroje, které jsem při vývoji využíval, a bude popsána struktura projektu. Následně budou přiblíženy vybrané části implementace a nakonec budou diskutovány další možná rozšíření.

\section{Použité nástroje}
Při implementaci jsem využíval několik nástrojů, které zefektivňují práci při vytváření a následných úpravách aplikací. Prvním takovým nástrojem je verzovací systém Git. Jde o svobodný a otevřený software, který umožňuje zaznamenávání jednotlivých revizí během vývoje aplikací. Jeho síla se nejvíce projeví v případě, kdy na projektu pracuje více vývojářů a je potřeba nějakým způsobem synchronizovat jejich práci. Pro své vlastnosti je využíván i největšími technologickými společnostmi jako jsou Google, Facebook nebo Microsoft. \cite{git}

Git byl na tomto projektu využit ve spojení s webovou službou GitHub, která nabízí bezplatné umístění repozitáře spolu s dalšími souvisejícími funkcemi jako je ticketovací systém nebo~možnost tvorby  stránek s dokumentací. Právě i na webové službě GitHub budou po obhájení práce dostupné kompletní zdrojové kódy aplikace pro případné zájemce\footnote{Repozitář se zdrojovými kódy aplikace bude dostupný na adrese: \url{https://github.com/paukert/kuoris-v2}}.

Dalším využitým nástrojem je program PhpStorm od firmy JetBrains. Jedná se o chytrý editor zdrojového kódu, který však obsahuje spoustu dalších podpůrných funkcí, které zjednodušují vývoj webových aplikací. Mezi tyto funkce se například řadí analýza zdrojového kódu, podpora pro ladění a testování a spousta dalších.


\section{Struktura projektu}
Projekt následuje standardní adresářovou strukturu Symfony aplikací a tedy orientace by lidem, kteří se již s nějakou aplikací vytvořenou v Symfony frameworku setkali, neměla dělat problém.

\dirtree{%
    .1 .
    .1 config\DTcomment{konfigurační soubory aplikace}.
    .1 public\DTcomment{kaskádové styly, kód v jazyku JavaScript, ikony}.
    .1 src.
    .2 Controller\DTcomment{controllery}.
    .2 Form\DTcomment{formuláře}.
    .2 Entity\DTcomment{modely}.
    .1 templates\DTcomment{šablony stránek}.
    .1 composer.json\DTcomment{soubor se seznamem závislostí}.
}

\section{Ukázky vybraných částí}
V následující sekci je popsáno několik vybraných částí z implementované aplikace, které jsou pro~vytvořený systém důležité nebo jsou nějakým způsobem zajímavé. Jako první bude představen způsob komunikace s informačním systémem Českého svazu orietačních sportů ORIS, poté bude popsána tvorba formuláře pro přidání nové události a jako poslední bude demonstrován způsob zobrazení aplikace na různých typech klientských zařízení.

\subsection{Komunikace s IS ORIS}

\subsection{Formuláře}
Jednou z komplikovanějších částí implementace bylo vytvoření formulářů pro přidávání a úpravu událostí. Symfony poskytuje propracovaný systém pro práci s formuláři, který zahrnuje jejich vytváření, renderování, zpracování nebo i validaci. Součástí zpracování formuláře je i možnost jeho propojení s konkrétním objektem, díky čemuž se vyplněné hodnoty do tohoto objektu automaticky propisují. Validace vyplněných hodnot probíhá jak na straně klienta díky nastavenému atributu \mintinline{text}|type| u tagu \mintinline{text}|input|, tak na straně serveru, kde je řešena s využitím Symfony validátorů, konkrétně pomocí „Symfony Validation Constraints“.

Pro vlastnosti objektu \mintinline{text}|Event|, jež jsou typu \mintinline{text}|string|, \mintinline{text}|integer| nebo \mintinline{text}|datetime|, nebylo vytváření formuláře nijak speciální. Zajímavější bylo vymyslet, jakým způsobem do formuláře přidat vlastnosti \mintinline{text}|categories| a \mintinline{text}|organizers|, která jsou v databázi reprezentované vazbou M:N mezi entitami \mintinline{text}|Event| a \mintinline{text}|Category|, respektive \mintinline{text}|Event| a \mintinline{text}|Organizer|. Uvedený typ vazby M:N v případě entit \mintinline{text}|Event| a \mintinline{text}|Category| znamená, že událost může mít libovolné množství kategorií a naopak konkrétní kategorie může být přiřazena k libovolnému počtu událostí.

Po zvážení dostupných možností jsem se rozhodl vytvořit formulář, ve kterém je možné dynamicky přidávat a odebírat nové kategorie a organizátory. Nové položky lze do formuláře přidat vybráním již existujícího záznamu registrovaného u jiné události, nebo vytvořením zcela nové položky. Finální vzhled části formuláře obsahující sekci s kategoriemi je zobrazen na obrázku \ref{figure:form}.

\begin{figure}[h]
    \caption{Část formuláře na přidání události}
    \label{figure:form}
    \centering
    \includegraphics[width=0.95\linewidth]{images/form.pdf}
\end{figure}

Logiku ukládání a odebírání položek nebylo třeba vymýšlet od začátku, poněvadž Symfony formuláře obsahují podporu pro přidávání libovolného počtu položek ke kontrétní vlastnosti. Pro~zprovoznění této funkcionality však bylo nezbytné přidat vlastní JavaScriptový kód, který zařídí samotné přidávání a odebírání položek z objektového modelu dokumentu (DOM). V souvislosti s dynamickým přidáváním nových položek bylo nutné navíc zamezit jejich duplikování v databázi v případě shody hodnot jejich vlastností.

\subsection{Responzivní design}
Jednou z důležitých vlastností nové aplikace je nepochybně responzivní design, díky kterému je vzhled systému optimalizován pro klientské zařízení s různým rozlišením. S ohledem na skutečnost, že v dnešní době přistupuje na web nejvíce lidí z mobilních zařízení \cite{deviceusage}, je tato vlastnost o to zásadnější. Na obrázcích \ref{figure:homepage-responsive-layout} a \ref{figure:form-responsive-layout} je vidět způsob zobrazení hlavní stránky a stránky s formulářem pro přidání události na chytrém telefonu.

Vytvořený responzivní vzhled stojí z velké části na frameworku Bootstrap, který již byl představen v sekci \ref{bootstrap}. Příkladem části, jež Bootstrap kompletně neřeší, jsou tabulky. Pro ty byl s využitím JavaScriptu a CSS vytvořen mechanizmus, který u zařízení s menším rozlišením postupně skrývá jednotlivé sloupce a naopak umožňuje zobrazení informací z těchto sloupců „rozbalením“ odpovídajícího řádku. Tato funkcionalita je demonstrována na obrázku \ref{figure:homepage-responsive-layout}.

\begin{figure}[h]
    \centering
    \begin{minipage}[b]{0.47\linewidth}
        \caption{Hlavní stránka}
        \label{figure:homepage-responsive-layout}
        \includegraphics[width=\linewidth]{images/homepage-responsive-layout.pdf}
    \end{minipage}
    \hfill
    \begin{minipage}[b]{0.47\linewidth}
        \caption{Formulář pro přidání události}
        \label{figure:form-responsive-layout}
        \includegraphics[width=\linewidth]{images/form-responsive-layout.pdf}
    \end{minipage}
\end{figure}

\section{Možná rozšíření}

Výslednou aplikaci lze v budoucnu rozšířit a vylepšit mnoha způsoby. Jeden z velkých nedostatků se týká evidence stavu klubových kont. Dnes je jejich stav po určitém časovém období zasílán vedoucím klubu na e-mailovou adresu ve formě XLSX souboru. V současné verzi aplikace je následně nutné manuálně zadat hodnotu klubového konta u každého ze členů dle zaslaného souhrnu. Tento nedostatek by mohl být například napraven možností importu stavu klubových kont v CSV formátu. Jako další řešení by se nabízelo automatické odečítání ze zůstatku konta v závislosti na evidovaných přihláškách, avšak tento přístup by stále vyžadoval ruční zásahy ve~chvíli, kdy si člen chce přidat prostředky na své osobní konto.

Mezi další možná vylepšení se jistě může zařadit zasílání informačních e-mailů při přidávání nových událostí, zobrazování statistik pro trenéry a administrátory nebo pokročilejší propojení s IS ORIS. Konkrétně by například mohla být přidána možnost výběru přihlášek, které se mají do~celorepublikového systému odeslat, nebo podpora pro~jejich úpravu a odstranění. V neposlední řadě by bylo dobré systém připravit pro lokalizaci do dalších jazyků.

