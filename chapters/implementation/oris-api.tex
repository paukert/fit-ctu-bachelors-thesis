\subsection{Komunikace s IS ORIS}
Jednou z nových funkcí oproti současně používanému systému je napojení na IS ORIS. Konkrétně je v nové aplikaci umožněno automatické načítání informací o závodu evidovaného v celorepublikovém systému a také odesílání evidovaných přihlášek. Propojení je dosaženo s využitím dostupného API, které je popsáno v sekci \ref{implementation:oris-api}.

\subsubsection{Webové API}\label{implementation:api}
Zkratku API lze do češtiny přeložit jako „rozhraní pro programování aplikací“. Z tohoto slovního spojení však člověk, který se v oblasti informačních technologií nepohybuje, pravděpodobně stále nebude mít moc představu, k čemu to API vlastně je. V případě webových aplikací se jedná o rozhraní, které umožňuje komunikaci mezi různými systémy nebo různými částmi systému. Jedna strana musí toto rozhraní, jež se může skládat z více koncových bodů (endpointů), poskytovat a druhá následně může na některý z těchto endpointů posílat povolené instrukce, a tím s první interagovat. V dnešní době je pro API orientovaná na data nejčastěji využívána architektura REST, avšak informační systém ORIS své API založil na vlastní architektuře, jejíž popis je součástí sekce \ref{implementation:oris-api}. \cite{api}

\subsubsection{HTTP}\label{http}
S webovými API je úzce spojen protokol HTTP. Jedná se o protokol aplikační vrstvy, který byl navržen pro komunikaci mezi webovým serverem a webovým prohlížečem. Je založen na~principu klient--server, kdy klient iniciuje vznik spojení za účelem zaslání požadavku a následně čeká na~odpověď od serveru. Každý HTTP požadavek mimo dalších informací obsahuje i název metody, která může specifikovat, jakým způsobem bude požadavek zpracován. Mezi nejdůležitější metody patří metoda GET, která slouží pro získání dat, a metoda POST, jež se nejčastěji využívá při~odesílání dat. \cite{http}

\subsubsection{ORIS API}\label{implementation:oris-api}
I přes zvyklosti uvedené v sekci \ref{http} probíhá komunikace s ORIS API pouze pomocí GET požadavků. Každý požadavek musí obsahovat query parametr s názvem metody a názvem formátu, ve kterém chceme data obdržet. V závislosti na vybrané metodě může být nutné uvést i další query parametry, které jsou pro~ni specifické. V současné době je podporováno 28 různých metod a 2 datové formáty (JSON a XML). Příklad požadavku na získání všech klubů v Českém svazu orientačních sportů ve formátu JSON můžeme vidět na výpisu kódu \ref{listing:api}. 

\begin{listing}[h]
    \caption{Požadavek na získání všech klubů v ČSOS}\label{listing:api}
    \begin{minted}{http}
GET /API/?format=json&method=getCSOSClubList HTTP/1.1
Host: oris.orientacnisporty.cz
Accept: */*
    \end{minted}
\end{listing}

V nové aplikaci jsou využity tři z nabízených metod. Konkrétně se jedná o metody \mintinline{text}|getEvent|, \mintinline{text}|getEventList| a \mintinline{text}|createEntry|. Metoda \mintinline{text}|getEvent| se využívá pro načtení informací o závodě při~vytváření nové události, \mintinline{text}|getEventList| se volá při hromadné kontrole času uzávěrky přihlášek a metoda \mintinline{text}|createEntry| se podílí na~odesílání přihlášek do IS ORIS.

\subsubsection{Importování nového závodu}
Jak již bylo zmíněno, při přidávání nového závodu je možné využít automatické načítání informací z celorepublikového systému. Procesy vykonávané v rámci této funkcionality lze rozdělit na~4~logické fáze:
\begin{enumerate}
    \item Získání ORIS ID závodu od uživatele
    \item Odeslání požadavku na ORIS API
    \item Zpracování přijatých dat
    \item Načtení zpracovaných dat do formuláře
\end{enumerate}

V rámci první fáze uživatel vyplní vstupní pole pro ORIS ID ve formuláři a klikne na tlačítko \emph{Importovat závod}. Kliknutím na uvedené tlačítko se spustí mechanismus, který na ORIS API odešle požadavek na zaslání dat týkající se závodu s odpovídajícím ORIS ID. Příklad validního požadavku je na výpisu kódu \ref{listing:get-event-request} a následná odpověď na výpisu kódu \ref{listing:get-event-response}.

\begin{listing}[h]
    \caption{Požadavek na získání informací o závodu}\label{listing:get-event-request}
    \begin{minted}{http}
GET /API/?format=json&method=getEvent&id=6734 HTTP/1.1
Host: oris.orientacnisporty.cz
Accept: */*
    \end{minted}
\end{listing}
\vspace{-3mm}

\begin{listing}[h]
    \caption{Začátek odpovědi na požadavek na získání informací o závodu}\label{listing:get-event-response}
    \begin{minted}{http}
HTTP/1.1 200 OK
Date: Sun, 17 Apr 2022 12:06:08 GMT

    \end{minted}
    \vspace{-11mm}
    \begin{minted}{json}
{
    "Method": "getEvent",
    "Format": "json",
    "Status": "OK",
    "ExportCreated": "2022-04-17 14:06:09",
    "Data": {
        "ID": "6734",
        "Name": "Oblastní žebříček",
        "Date": "2022-04-02",
        "Place": "Lomnice nad Popelkou",
        "Map": "Tábor, 1:10000",
    \end{minted}
\end{listing}

Z přijatých dat je následně pomocí speciální třídy sestaven nový objekt typu \mintinline{text}|Race|, jenž je poté předán formuláři. Ten si již jen načte data z tohoto objektu a výsledná stránka včetně předvyplněných hodnot je zaslána uživateli.

\newpage

\subsubsection{Odesílání přihlášek}
Pro odeslání přihlášek do IS ORIS je zapotřebí poslat GET požadavek, který kromě názvu metody a formátu dat obsahuje i query parametry \mintinline{text}|username|, \mintinline{text}|password|, \mintinline{text}|class| a \mintinline{text}|clubuser|. Takovýto požadavek je potřeba poslat pro každou evidovanou přihlášku, poněvadž systém ORIS nepodporuje posílání informací o více přihláškách najednou.

Parametry \mintinline{text}|username| a \mintinline{text}|password| slouží k autorizaci požadavku a jedná se o jméno a heslo, pomocí kterých uživatel přistupuje do IS ORIS. Tento způsob autentizace však není doporučovaný z několika důvodů. Prvním z nich je skutečnost, že na straně serveru bývá požadovaná URL včetně query parametrů ukládána do přístupových záznamů. Ke stejné situaci dochází i na straně klienta, kde je URL ukládána do historie prohlížeče. Další problém může vzniknout, pokud by stránky využívaly JavaScript, CSS nebo jiné zdroje třetích stran, protože požadavek na tyto zdroje může obsahovat hlavičku \mintinline{text}|Referer| s kompletní URL. Ani jedna z těchto vlastností není žádoucí, a proto by se měly využívat jiné dostupné možnosti autentizace. \cite{query-parameters-security}

Dalším povinným parametrem je \mintinline{text}|class|, který představuje identifikátor kategorie v IS ORIS a který je uložen v databázi vytvořené aplikace v entitě \mintinline{text}|Category|. Posledním povinným query parametrem je \mintinline{text}|clubuser|, jenž identifikuje přihlašovaného uživatele. I tento identifikátor byl nakonec uložen do databáze vytvořeného systému, poněvadž pro jeho získání by bylo potřeba poslat dva další požadavky. Pro každou přihlášku by tedy bylo potřeba celkově poslat tři požadavky místo jednoho. Příklad požadavku na vytvoření přihlášky pro člena klubu s ID \emph{12007} do~kategorie \emph{156818} je na výpisu kódu \ref{listing:create-entry}.

\begin{listing}[h]
    \caption{Požadavek na vytvoření přihlášky}\label{listing:create-entry}
    \begin{minted}[breaklines, breakbefore=&]{http}
GET /API/?format=json&method=createEntry&username=Paukert&password=XXXX&clubuser=12007&class=156818 HTTP/1.1
Host: oris.orientacnisporty.cz
Accept: */*
    \end{minted}
\end{listing}
\vspace{-6mm}
