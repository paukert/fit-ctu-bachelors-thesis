\section{Použité nástroje}
Při implementaci jsem využíval několik nástrojů, které zefektivňují práci při vytváření a následných úpravách aplikací. Prvním takovým nástrojem je verzovací systém Git. Jde o svobodný a otevřený software, který umožňuje zaznamenávání jednotlivých revizí během vývoje aplikací. Jeho síla se nejvíce projeví v případě, kdy na projektu pracuje více vývojářů a je potřeba nějakým způsobem synchronizovat jejich práci. Pro své vlastnosti je využíván i největšími technologickými společnostmi jako jsou Google, Facebook nebo Microsoft. \cite{git}

Git byl na tomto projektu využit ve spojení s webovou službou GitHub, která nabízí bezplatné umístění repozitáře spolu s dalšími souvisejícími funkcemi jako je ticketovací systém nebo~možnost tvorby  stránek s dokumentací. Právě i na webové službě GitHub budou po obhájení práce dostupné kompletní zdrojové kódy aplikace pro případné zájemce\footnote{Repozitář se zdrojovými kódy aplikace bude dostupný na adrese: \url{https://github.com/paukert/kuoris-v2}}.

Dalším využitým nástrojem je program PhpStorm od firmy JetBrains. Jedná se o chytrý editor zdrojového kódu, který však obsahuje spoustu dalších podpůrných funkcí, které zjednodušují vývoj webových aplikací. Mezi tyto funkce se například řadí analýza zdrojového kódu, podpora pro ladění a testování a spousta dalších.
