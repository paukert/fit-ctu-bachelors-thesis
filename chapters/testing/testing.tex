\chapter{Uživatelské testování}
Tato kapitola se zabývá testováním vytvořeného prototypu aplikace. Jak už název napovídá, tak při uživatelském testování ověřují použitelnost aplikace samotní uživatelé. Hlavním cílem je zjistit, zda je uživatelské rozhraní dostatečně intuitivní, a získat případné návrhy na~jeho vylepšení.

Testování proběhlo za osobní přítomnosti a zúčastnily se ho celkem 4 osoby včetně předsedy klubu KOB~Ústí nad Orlicí. Své zastoupení zde měly různé generace a byli přizváni jak uživatelé současného systému, tak i lidé, kteří s ním nikdy nepracovali. Každý tester byl na začátku seznámen s podstatou systému a následně obdržel sadu úkolů ze sekce \ref{test-cases}, které za dozoru postupně procházel.

\section{Testovací scénáře}\label{test-cases}

\subsection{Registrace a přihlášení do systému}\label{test-case-1}
\begin{description}
    \item[Modelová situace] \hfill \\
        Představte si, že jste nový člen klubu a chcete získat přístup do webové aplikace.
    \item[Postup] \hfill
        \vspace{-2mm}
        \begin{enumerate}
            \item Přejděte na stránku s registračním formulářem, vyplňte požadované údaje a klikněte na~tlačítko \emph{Registrovat}.
            \item Vyčkejte na schválení vytvořeného účtu administrátorem.
            \item Pomocí dříve zadaných údajů se přihlaste do aplikace.
        \end{enumerate}
\end{description}

\subsection{Změna kontaktního e-mailu a hesla}\label{test-case-2}
\begin{description}
    \item[Modelová situace] \hfill \\
        Na váš oblíbený internetový obchod zaútočili hackeři a ukradli údaje ke stovkám účtů. Bohužel e-shop využíval k hashování hesel zastaralý mechanismus a byl vám z tohoto důvodu ukraden i e-mail, ke kterému jste používali stejné heslo. Nyní si chcete v systému nastavit novou e-mailovou adresu a změnit přihlašovací heslo.
        \item[Postup] \hfill
        \vspace{-2mm}
        \begin{enumerate}
            \item Přejděte pomocí hlavní nabídky na stránku \emph{Nastavení}.
            \item Zadejte svou novou e-mailovou adresu a heslo a změny uložte.
        \end{enumerate}
\end{description}

\newpage
\subsection{Přidání nového závodu a jeho následná úprava}\label{test-case-3}
\begin{description}
    \item[Modelová situace] \hfill \\
        Představte si, že jste trenér klubu KOB Ústí nad Orlicí a chcete přidat do systému nový závod Východočeského poháru.
    \item[Postup] \hfill
        \vspace{-2mm}
        \begin{enumerate}
            \item Přejděte do administrace systému a zvolte možnost pro přidání nového závodu.
            \item Využijte možnosti načtení informací o závodu ze systému Českého svazu orientačních sportů ORIS, ve kterém má závod přidělený identifikátor \emph{6738}.
            \item Přidejte tuto událost se všemi načtenými informacemi a názvem \emph{7. kolo VčP} do systému.
            \item Na následně zobrazené stránce jste si všimli chybného data uzávěrky přihlášek, vraťte se do~administrace a změňte ho na \emph{7. 5. 2022 23:59}.
        \end{enumerate}
\end{description}

\subsection{Přihlášení na závod a napsání komentáře}\label{test-case-4}
\begin{description}
    \item[Modelová situace] \hfill \\
        Jako člen klubu se chcete přihlásit na nadcházející závod Východočeského poháru a napsat k němu komentář.
    \item[Postup] \hfill
        \vspace{-2mm}
        \begin{enumerate}
            \item Z hlavní stránky systému nebo ze seznamu všech závodů si zobrazte detailní informace o závodu s názvem \emph{7. kolo VčP}.
            \item Vyberte si požadovanou kategorii a přihlášení potvrďte.
            \item Napište k této události libovolný komentář.
        \end{enumerate}
\end{description}

\section{Průběh a výsledky}
První modelová situace ze scénáře \ref{test-case-1} se zabývá registrací a přihlášením do systému. Všichni testeři se pomocí položky v hlavní nabídce případně s využitím tlačítka na uvítací obrazovce bez~problému dostali na stránku s registračním formulářem. Samotný proces registrace do systému a i následného přihlášení považovali respondenti za jednoduchý.

Scénář \ref{test-case-2} měl za úkol ověřit, zda uživatelé bez potíží najdou sekci s nastavením, kde si mohou změnit některé ze svých údajů. Bod ve scénáři zmiňuje, aby přešli pomocí hlavní nabídky na stránku \emph{Nastavení}. Jeden z testerů, jehož považuji za průměrně zdatného uživatele počítače, chvilku přemýšlel, jak se do nastavení dostat, nicméně i jemu se to nakonec podařilo. Pro~rozbalovací menu byla z tohoto důvodu nakonec zvolena jiná ikona, kterou oslovení respondenti hodnotí jako názornější.

Přidání a následná úprava závodu byly testovány v rámci scénáře \ref{test-case-3}. Rozložení administrace přišlo respondentům přehledné a s vyhledáním možnosti pro přidání nového závodu neměl nikdo problém. Načtení informací o události z IS ORIS nedělalo uživatelům taktéž obtíže a i následná úprava závodu není dle jejich slov složitá.

Poslední modelová situace ze scénáře \ref{test-case-4} ověřovala intuitivnost uživatelského rozhraní při~přihlašování se na událost. Přechod na stránku s detailními informacemi o závodu nečinil potíže žádnému z respondentů. Při samotném přihlašování na událost jeden z testerů navrh, aby se mu v nabídce zobrazovaly pouze kategorie, jež jsou pro něj relevantní (tj. aby se mužům nezobrazovaly kategorie pro ženy a aby proběhlo filtrování na základě věku přihlašované osoby). Tuto funkcionalitu jsem sám plánoval v budoucnu naimplementovat s využitím ORIS API, které nabízí metodu \mintinline{text}|getValidClasses| vracející seznam platných kategorií pro konkrétního uživatele a závod. Kromě této připomínky hodnotili respondenti proces přihlašování a psaní komentářů jako jednoduchý.

Aplikaci oslovení testeři až na dvě výjimky popsané výše hodnotili jako přehlednout a uživatelsky přívětivou. Jedna ze dvou připomínek již byla zapracována a druhá bude vyřešena v rámci budoucího vylepšení integrace se systémem Českého svazu orientačních sportů ORIS.
