\chapter{Závěr}

Cílem práce bylo navrhnout a implementovat prototyp webové aplikace, která bude sloužit sportovním klubům, a to především těm, které se věnují orientačnímu běhu.

Na základě cílů práce byla nejprve provedena analýza, která zahrnovala popis již existujících řešení a specifikaci funkčních a nefunkčních požadavků. Následoval návrh nové aplikace a výběr vhodných technologií. S ohledem na omezení, která byla stanovena v rámci nefunkčních požadavků, byl prototyp řešení vyvinut v programovacím jazyku PHP s využitím frameworku Symfony.

Aplikace umožňuje registraci a správu uživatelů a evidenci událostí, na které je možné se přihlašovat a psát k nim komentáře. Díky propojení s informačním systémem Českého svazu orientačních sportů ORIS je umožněno závody, které jsou v něm evidované, do vytvořené aplikace jednoduše importovat. Do systému ORIS z ní naopak lze odeslat všechny přihlášky evidované u konkrétní události.

Vytvořený prototyp webové aplikace byl uživatelsky otestován a na základě výsledků byly navrženy drobné úpravy. Systém byl zároveň připraven pro nasazení místo současně používané aplikace.

Všechny cíle této práce byly splněny. Zdrojové kódy výsledné aplikace budou veřejně přístupné na internetové službě GitHub a kdokoliv bude mít možnost si vytvořené řešení stáhnout a případně upravit podle svých potřeb. Osobně plánuji ve vývoji nadále pokračovat i po odevzdání této práce. Vytvořený prototyp je možné například vylepšit o možnost importování stavu osobních kont, zobrazování statistik pro trenéry a administrátory nebo o větší integraci se systémem ORIS.
