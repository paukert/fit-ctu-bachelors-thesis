\chapter{Závěr}

Cílem práce bylo navrhnout a implementovat prototyp webové aplikace, jež bude sloužit sportovním klubům, a to především těm, které se věnují orientačnímu běhu.

Na základě cílů práce byla nejprve provedena analýza, která zahrnovala popis již existujících řešení a specifikaci funkčních a nefunkčních požadavků. Následoval návrh nové aplikace a výběr vhodných technologií. S ohledem na omezení, jež byla stanovena v rámci nefunkčních požadavků, byl systém vyvinut v programovacím jazyku PHP s využitím frameworku Symfony.

Vytvořená aplikace umožňuje registraci a správu uživatelů a evidenci událostí, na které je možné se přihlašovat a psát k nim komentáře. S využitím dostupného API informačního systému Českého svazu orientačních sportů byl vytvořen mechanismus pro snadné importování závodů do nově naprogramované aplikace. Do systému ORIS lze naopak z vyvinutého systému odeslat všechny přihlášky evidované u konkrétní události.

Po dokončení implementace byla webová aplikace podrobena uživatelskému testování a na~základě výsledků byly navrženy drobné úpravy. Systém byl zároveň připraven pro nasazení místo aktuálně využívané aplikace, kterou by jistě mohl již v současném stavu nahradit.

Všechny cíle této práce byly splněny. Zdrojové kódy výsledné aplikace budou veřejně přístupné na internetové službě GitHub a kdokoliv bude mít možnost si vytvořené řešení stáhnout a případně upravit podle svých potřeb. Osobně plánuji ve vývoji nadále pokračovat i po odevzdání této práce. Vytvořenou aplikaci je možné například vylepšit o možnost importování stavu osobních kont, zasílání informačních e-mailů při přidávání nových událostí, zobrazování statistik pro trenéry a administrátory nebo o větší integraci s informačním systémem ORIS.
